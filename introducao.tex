\section{Introdução}

\frame
{
\frametitle{Introdução}
\begin{itemize}
\item Situação do trânsito grave nas grandes cidades.
\item Projeto de lei para priorizar o uso do transporte público e não motorizado.
\item A cada 12 reais investidos no transporte particular, somente 1 é investido no transporte público.
\end{itemize}

\begin{table}[!htb]
	\scriptsize
	\centering
	\caption{Relação entre carros e ônibus nas principais cidades brasileiras}
	\label{tab:carro_onibus}
	\begin{tabular}{lcccc}
		\hline
		\textbf{Cidade} & \textbf{\% Carros} & \textbf{\% Ônibus} \\
		\hline
		\textbf{Belo Horizonte} & 77 & 23 \\
		\textbf{Brasília} & 91 & 9 \\
		\textbf{Curitiba} & 79 & 21 \\
		\textbf{Porto Alegre} & 69 & 31 \\
		\textbf{Recife} & 84 & 16 \\
		\textbf{Rio de Janeiro} & 74 & 26 \\
		\textbf{São Paulo} & 88 & 12 \\
		\hline
	\end{tabular}
\end{table}
}

\subsection{Motivação}
\frame
{
\frametitle{Motivação}
\begin{itemize}
\item Desconhecimento das linhas de transporte presentes em um determinado centro urbano por parte dos usuários.
\item Sistemas de transporte mais complexos.
\item Canal de comunicação entre usuários do transporte público e responsáveis por sua disponibilidade.
\item Evolução das ferramentas computacionais.
\end{itemize}
}

\subsection{Objetivos}
\frame
{
\frametitle{Objetivos}
	
\begin{itemize}
\item Desenvolver uma ferramenta capaz de auxiliar o planejamento de trajetos urbanos com o uso de transporte público.
\item Desenvolver uma interface na forma de sítio eletrônico para os usuários do serviço.
\item Distribuir as tecnologias desenvolvidas sob a forma de \emph{software livre}.
\end{itemize}
}

\subsection{Justificativa}
\frame
{
\frametitle{Justificativa}
Integração e aplicação de conteúdos estudados ao longo do curso para aquisição de experiência em desenvolvimento de projetos:
\begin{itemize}
	\item Sistemas distribuídos.
	\item Desenvolvimento para \emph{web}.
	\item Algoritmos.
	\item Estruturas de dados.
	\item Bancos de dados.
	\item Engenharia de \emph{software}.
\end{itemize}



}
