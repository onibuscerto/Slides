\section{Metodologia}

\frame
{
\frametitle{Cenário}
\begin{itemize}
	\item Equipe de três integrantes.
	\item Diferente disponibilidade de horários.
	\item Diferentes conjuntos de habilidades.
	\item Complexidade do projeto.
\end{itemize}
}

\subsection{Processo de Desenvolvimento do Software}
\frame
{
\frametitle{Processo de Desenvolvimento do Software}
\begin{itemize}
	\item Metodologia inspirada em Métodos Ágeis:
	\begin{itemize}
		\item \emph{Software} funcional.
		\item Documentação sucinta.
		\item Desenvolvimento orientado a testes.
	\end{itemize}
	\item Reuniões periódicas:
	\begin{itemize}
		\item Controle de qualidade.
		\item Revisão de código.
		\item Estabelecimento de novas metas.
		\item Alocação das tarefas.
	\end{itemize}
\end{itemize}
}
\frame
{
\frametitle{Desenvolvimento Orientado a Testes}
\begin{itemize}
	\item Ciclo original:
	\begin{enumerate}
		\item Adicionar teste.
		\item Executar testes e verificar falhas.
		\item Escrever código e corrigir falhas.
		\item Refatorar código.
		\item Repetir.
	\end{enumerate}
	\item Adaptação da metodologia.
	\item Vantagens:
	\begin{itemize}
		\item Passos menores.
		\item Cumprimento de metas.
		\item Minimiza os \emph{bugs}.
		\item Mantém a simplicidade do código.
	\end{itemize}
\end{itemize}
}
\frame
{
\frametitle{Revisão de Código}
\begin{itemize}
	\item
\end{itemize}
}

\subsection{Ferramentas Utilizadas}
\frame
{
\frametitle{Ferramentas Utilizadas}
\begin{itemize}
	\item oi
\end{itemize}
}
