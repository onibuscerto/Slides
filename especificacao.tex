\section{Requisitos do Sistema}

\subsection{Requisitos Funcionais}
\frame
{
\frametitle{Requisitos Funcionais}
Quanto à interface do sistema:
\begin{itemize}
	\item Rota com menor tempo de viagem.
	\item Rota resultante como sequência de passos e desenhada em um mapa.
	\item Entrada no formato de endereço ou marcada no mapa.
	\item Navegador web.
\end{itemize}
}

\subsection{Requisitos Não-Funcionais}
\frame
{
\frametitle{Requisitos Não-Funcionais}
Quanto a distribuição do \emph{software}:
\begin{itemize}
	\item Licença livre.
	\item Serviço web.
\end{itemize}
}

\frame
{
\frametitle{Requisitos Não-Funcionais}
Quanto à fonte de dados:
\begin{itemize}
	\item Independente de fonte de dados (padrão GTFS).
	\item Banco de dados de grafos.
	\item Mínimo de informações sobre as rotas do serviço de transporte público, sendo estas:
	\begin{itemize}
		\item As rotas existentes.
		\item As paradas ou estações existentes.
		\item Horários de chegada e partida de veículos de cada parada ou estação.
		\item As coordenadas geográficas de cada parada ou estação.
	\end{itemize}
\end{itemize}
}

\frame
{
\frametitle{Requisitos Não-Funcionais}
Quanto às ferramentas:
\begin{itemize}
	\item Suporte do SGBD a consultas rápidas.
	\item Hospedagem em um servidor que suporte todas as ferramentas computacionais utilizadas.
\end{itemize}
}