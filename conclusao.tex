\section{Considerações Finais}

\frame
{
\frametitle{O Sistema}
\begin{itemize}
	\item Metodologia de desenvolvimento distribuído possível graças a modularização do sistema.
	\item Auxílio de ferramentas.
	\item Sistema flexível quanto ao banco de dados.
	\item \emph{Log} de comentários dos desenvolvedores, o que facilita futuras contribuições.
\end{itemize}
}

\frame
{
\frametitle{Relação com Engenharia de Computação}
Experiência agregada pela equipe em desenvolvimento de projetos com integração de conceitos aprendidos ao longo do curso.
\begin{itemize}
\item Desenvolvimento \emph{web}.
\item Sistemas Distribuídos.
\item Teoria dos Grafos.
\item Algoritmos.
\item Estrutura de dados.
\item Banco de dados.
\item Engenharia de Software.
\end{itemize}
}

\frame
{
\frametitle{Propostas Futuras}
\begin{itemize}
\item Otimização do sistema.
\item Diferentes clientes.
\item Mais informações dos arquivos GTFS.
\item \emph{Shapes}.
\item \emph{Layout} da interface \emph{web}.
\item Novas funcionalidades como busca baseada no horário de chegada ao destino.
\item Busca multi-objetivo.
\item Dados de Curitiba e outras cidades brasileiras.
\end{itemize}
}

\frame
{
\frametitle{Perguntas?}
\begin{center}
\vskip15pt \Huge Obrigado! \\
~ \\
\normalsize \texttt{http://github.com/onibuscerto/} \\
~ \\
brunoweingraber@gmail.com \\
lucascpaiva@gmail.com \\
luizribeiro@gmail.com \\
\end{center}
}